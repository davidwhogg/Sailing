% This paper is copyright 2021 the authors.

% people who might comment on it:
% -------------------------------
% - KSF
% - Kilian Walsh

% to-do items:
% ------------
% - FIX THE FORCE EXPRESSION and discuss it in terms of inner products, tensors, and Einstein summation notation.
% - Put in running head.
% - Overleaf seems to have wrong papersize?
% — Put in the dumbell figure
% - Reconsider the post-equation commas? (HWR SAY).
% — Find references for, and destroy, Bernoulli.
% — Make some comments on the situations in which the ram-pressure force approximation is more appropriate---gale force or lull?

\documentclass{article}
\usepackage[utf8]{inputenc}

% math macros
\usepackage{amsmath, amsfonts}
\DeclareMathOperator*{\argmax}{argmax}
\newcommand{\dd}{\mathrm{d}}
\renewcommand{\vec}[1]{\boldsymbol{#1}}
\newcommand{\uvec}{\vec{\hat{e}}}
\newcommand{\tensor}[1]{\mathbf{#1}}
\newcommand{\air}{\text{air}}
\newcommand{\water}{\text{wat}}
\newcommand{\boat}{\text{boat}}
\newcommand{\destination}{\text{dest}}
\newcommand{\good}{\text{(good)}}
\newcommand{\best}{\text{(best)}}
\newcommand{\sail}{\text{s}}
\newcommand{\keel}{\text{k}}
\newcommand{\vair}{\vec{v}_\air}
\newcommand{\vwater}{\vec{v}_\water}
\newcommand{\vboat}{\vec{v}_\boat}
\newcommand{\vdest}{\vec{v}_\destination}

% text macros
\usepackage{biblatex}
\addbibresource{sailing.bib}
\newcommand{\documentname}{\textsl{Article}}
\newcommand{\secref}[1]{Section~\ref{#1}}

% typesetting adjustments
\makeatletter
\renewcommand\section{\@startsection {section}{1}{\z@}%
  {-3.25ex \@plus -1ex \@minus -.2ex}%
  {1.5ex \@plus .2ex}%
  {\raggedright\normalfont\large\bfseries}}
\makeatother
\setlength{\textwidth}{5.00in}
\setlength{\textheight}{9.00in}
\setlength{\topmargin}{-0.40in}
\pagestyle{myheadings}
\markright{\textsf{Hogg \& Kleban / Ram-pressure model for sailing}}
\linespread{1.08}
\frenchspacing\sloppy\sloppypar\raggedbottom

\title{\bfseries%
A simple, anisotropic ram-pressure model for sailing}
\author{DWH, MK, others?}
\date{June 2021}
\begin{document}
\maketitle

\begin{abstract}\noindent
    We present a simplified model for the forces on a sailboat.
    Our model is that there is a ram-pressure force on the sail, with a magnitude and direction that depend on the alignment of the relative air--boat velocity with the sail orientation according to an anisotropic drag tensor.
    Similarly, there is a ram-pressure force on the keel, depending on the relative water--boat velocity.
    The resulting expression for the total force on the boat is coordinate-free and consistent with Galilean relativity.
    The solutions corresponding to sailing in steady wind correspond to boat velocities that deliver vanishing total force, given the angles of the sail and keel.
    We use this model to derive simple rules for sailing that are not far off from standard sailing practice, for sailing in different directions (including upwind).
    One consequence of this model is that a sailboat does not move (relative to the water) precisely in the direction that its keel points (the direction of the prow); it moves slightly downwind of that.
    Another consequence is that it is possible for an extremely anisotropic (extremely well designed) boat to sail downwind (relative to the water) \emph{faster than the wind}, on a tacking course.
    Interestingly, nothing about the mechanical model or discussion depend on the \emph{curvature} of the sail, in contradiction to some folk explanations of sailing.
\end{abstract}

\section{Introduction}\label{sec:intro}

KLEBAN: Write things about ddwfttw, maybe?

HOGG: Stuff about Bernoulli and flying and aerofoils that turned out to be wrong; a motivation to reconsider sailing too.

\section{Ram-pressure sailboat}\label{sec:boat}

We are going to work in 2D, where velocities $\vec{v}$ are column vectors (or $\vec{v}\in\mathbb{R}^{2\times 1}$).
To stay coordinate-free, we are going to talk about angles as orientations of unit vectors like $\uvec_{\perp\sail}$ and $\uvec_{\parallel\sail}$ (which will be defined below).
If you prefer to think in terms of angles, you can choose a coordinate system and then
\begin{align}
    \uvec_{\perp\sail} &= \begin{bmatrix}\cos\theta_\sail \\ \sin\theta_\sail\end{bmatrix} ~ ; ~ \uvec_{\parallel\sail} = \begin{bmatrix}-\sin\theta_\sail \\ \cos\theta_\sail\end{bmatrix} ~,
\end{align}
for some angle $\theta_\sail$ in that coordinate system.
We won't particularly use angles much in what follows, but if you want to come back to angles, you can take the arctangent of the components of $\uvec_{\perp\sail}$.

Our sailboat is going to be an anisotropic object such that wind that hits the sail perpendicular to the large surface of the sail (in the rest frame of the boat) will produce more ram pressure than wind at any other angle (and far more than wind that ``goes along'' the sail direction).
We can construct the complete force law on the sailboat by thinking about ram-pressure forces on surfaces.
If the fluid (air or water in this case) is thought of as being composed of particles that bounce elastically off of a flat surface of area $A$, then the ram-pressure force on that surface is
\begin{align}
    \vec{F} &= \eta\,\rho\,A\,(\uvec_\perp^\top\vec{v})^2\,\uvec_\perp ~,
\end{align}
where $\eta$ is a dimensionless constant (naively 2, but HOGG), $\rho$ is the density of the fluid, $\uvec_\perp$ is the unit vector perpendicular to the surface (oriented such that it points down-flow), $\vec{v}$ is the velocity of the fluid in the rest frame of the surface, and $\uvec_\perp^\top\vec{v}$ is the scalar inner product (dot product) of the two vectors.
If you like thinking about tensors, this can be thought of as a tensor expression, in which $\rho$, $A$, and the unit vectors are combined into a three-index (order-three) tensor $\mathbb{A}$, which is then contracted onto the velocity twice:
\begin{align}
    \mathbb{A} &\equiv \eta\,A\,\uvec_\perp\,\uvec_\perp^\top\otimes\uvec_\perp^\top ~.
\end{align}
In Einstein summation notation, the three-index tensor is contracted as follows
\begin{align}
    [\vec{F}]_i &= \rho\,[\mathbb{A}]_{ijk}\,[\vec{v}]_j\,[\vec{v}]_k ~,
    \\
    [\mathbb{A}]_{ijk} &= \eta\,A\,[\uvec_\perp]_i\,[\uvec_\perp]_j\,[\uvec_\perp]_k
    ~,
\end{align}
where the $[\vec{F}]_i$ notation indicates the $i$th component or coordinate of the vector, the tensor $\mathbb{A}$ has three indices, and the repeated indices are summed over.

HOGG: PARAGRAPH AND EQUATIONS ALL WRONG: In detail, the ram-pressure force $\vec{F}_\air$ from the air on the sailboat will be
HOGG THIS EXPRESSION IS ALL WRONG. THE FORCE IS GIVEN BY DOUBLE-CONTRACTION ON AN ORDER-2 TENSOR. SEE NOTES 2021-06-19. FIX THIS.
\begin{align}
    \Delta\vair &\equiv \vair-\vboat
    \\
    \vec{F}_\air &= \rho_\air\,|\Delta\vair|\,\tensor{A}_\air\,\Delta\vair \label{eq:Fair}
    \\
    \tensor{A}_\air &= A_{\perp\sail}\,\uvec_{\perp\sail}\,\uvec_{\perp\sail}^\top + A_{\parallel\sail},\uvec_{\parallel\sail}\,\uvec_{\parallel\sail}^\top \label{eq:Aair}
    \\
    A_{\perp\sail} &\gg A_{\parallel\sail} \label{eq:sail}
    ~,
\end{align}
where $\rho_\air$ is the mass density of air, $\vair,\vboat$ are the velocities of the air and boat, $\tensor{A}_\air$ is a Hermitian tensor with units of area, $A_{\perp\sail}, A_{\parallel\sail}$ are two constants with units of area defining the geometry of the sail, $\uvec_{\perp\sail}$ is a unit vector pointing normal to the surface of the (assumed planar) sail, and $\uvec_{\parallel\sail}$ is a unit vector perpendicular to that.
If you are uncomfortable with the expression for the tensor $\tensor{A}_\air$ in \eqref{eq:Aair}, look at the discussion of tensors in, for example, \cite{kusse}.
The inequality \eqref{eq:sail} says (loosely) that the sail (plus hull) is much bigger face-on than edge-on.
The force $\vec{F}_\water$ from the water is similar
\begin{align}
    \Delta\vwater &\equiv \vwater-\vboat
    \\
    \vec{F}_\water &= \rho_\water\,|\Delta\vwater|\,\tensor{A}_\water\,\Delta\vwater \label{eq:Fwater}
    \\
    \tensor{A}_\water &= A_{\perp\keel}\,\uvec_{\perp\keel}\,\uvec_{\perp\keel}^\top + A_{\parallel\keel},\uvec_{\parallel\keel}\,\uvec_{\parallel\keel}^\top
    \\
    A_{\perp\keel} &\gg A_{\parallel\keel} ~,
\end{align}
where we have gone from air to water and sail ``s'' to keel ``k''.
The ram-pressure forces given in \eqref{eq:Fair} and \eqref{eq:Fwater} can be justified with a simple molecular model in which air and water molecules are elastically bouncing off the sail and keel surfaces; this is the same set of assumptions that leads to, for example, the ideal gas law (HOGG CITE).

Mathematically, steady sailing in steady wind and steady current (constant $\vair, \vwater$) is given by
\begin{align}\label{eq:sailing}
    \vec{F} &= \vec{F}_\air + \vec{F}_\water = \vec{0} ~.
\end{align}
If we want to operationalize steady sailing in code, steady sailing can be thought of as a root-finding problem, in which we find the vector $\vboat$ that makes \eqref{eq:sailing} true, given the orientations $\uvec_{\perp\sail}, \uvec_{\perp\keel}$ of the sail and keel.
This is a quadratic root-finding problem so it has HOGG SOME SIMPLE CLOSED FORM, WHICH IS... we CAN ALSO MAYBE solve it numerically in the code associated with this \documentname\footnote{HOGG GIVE LINK HERE} by means of the vector generalization of Newton's method (we return to this in \secref{sec:implementation}).

HOGG: Give the settings of all our parameters for the figures we are about to show.

HOGG: Define direction cosines with the wind, and with boat parts, using vectors.

HOGG: PUT FIGURES HERE SHOWING THE BEHAVIOR OF THE BOAT IN THE WIND.

HOGG: Put the random-dot hourglass figure here.

HOGG: Comments: You can see from the figures that you don't sail DIRECTLY in the direction the keel is pointed.

HOGG: Right now, the concept of forwards and backwards is a bit soft; our boat is quadrupolar! But we will get more serious about deliberately sailing in the correct direction in the next Section.

HOGG: Comment: It is easy to sail downwind fttw.

HOGG: Comment: It is easy to sail faster than the wind speed.

HOGG: Comment: It is easy to sail upwind. But impossible to sail upwind fttw.

HOGG: Comment: You can't sail directly into the wind, or within XXX radians of directly into the wind. This XXX depends on dimensionless sail and boat ratios.

\section{Good sailing}\label{sec:sailing}

In \secref{sec:boat} we showed how the boat moves, given any arbitrary setting of the orientation of the keel and the sail.
However, one doesn't sail by randomly setting the keel and sail!
One sails by pointing the boat in some direction and then setting the sail (or sails) appropriately.
In our simple, ram-pressure boat, how should we set the planar sail?

HOGG: IN THIS SECTION EXPLICITLY DEFINE ``GOOD SAILING''.
\begin{align}\label{eq:good}
    \uvec_{\perp\sail}^\good &\leftarrow \argmax_{\uvec_{\perp\sail}} \left[\uvec_{\parallel\keel}^\top\,(\vboat-\vwater)\right] ~,
\end{align}
where .... HOGG IMPLICITLY $\vboat-\vwater$ is a function of all the properties of the world and the boat boat, along with the orientation of its keel and sail.

HOGG: FIGURE showing good sailing for a set of boat orientations.

HOGG: Figure showing direction cosines of sail relative to keel/boat orientation.

HOGG: Put the equivalent of the hourglass figure.

HOGG: Comment: The settings of the sail look ``tight'' relative to standard sailing practice. But this boat has flat sails, so it's hard to compare the exact settings with sailor experiences (which are all with curved sails).

HOGG: Final comment: But this is not the BEST a sailor can do...

\section{Competitive sailing}\label{sec:racing}

HOGG IN THIS SECTION EXPLICITLY DEFINE ``COMPETITIVE SAILING''.

A type-A sailor---or maybe a competetive sailor---is trying to get from a current position to a destination as quickly as possible.\footnote{We don't particularly endorse this attitude towards sailing. The journey \emph{is} the destination.}
Our view is that this sailor should start by specifying the unit vector $\uvec_\destination$ that points from the current position of the boat towards the destination of the boat.
The sailor should then maximize the scalar product $\uvec_\destination^\top\,\vboat$
\begin{align}\label{eq:best}
    \uvec_{\perp\sail}^\best,\uvec_{\perp\keel}^\best &\leftarrow \argmax_{\uvec_{\perp\sail},\uvec_{\perp\keel}} \left[\uvec_\destination^\top\,(\vboat-\vdest)\right] ~,
\end{align}
where $\uvec_{\perp\sail}^\best,\uvec_{\perp\keel}^\best$ are the best settings of the orientations of the keel and sail, given the direction $\uvec_\destination$ to the destination, and (because we are Galilean-relativistic) this velocity is relative to the velocity $\vdest$ of the destination (which might not be stationary with respect to the water).
In this expresion \eqref{eq:best}, implicitly $\vboat$ is being thought of as a function of these orientations (and more).
Note that the sailor \emph{does not necessarily want} to travel directly towards the destination:
The sailor wants to make as much progress as possible in that direction, but is willing to tack back-and-forth to get there, if it gets the boat there faster.
So the metric really is the projection of the boat velocity onto the displacement vector pointing from the current position of the boat to the destination.

While this is simple to state, the numerical optimization hurts...HOGG

HOGG PUT FIGURES HERE SHOWING THE OPTIMAL SETTINGS OF SAIL AND KEEL ORIENTATIONS FOR DIFFERENT DESTINATION DIRECTIONS.

Now if you want to plan a complete path or journey for the sailboat in (unrealistically) steady wind, you... HOGG

HOGG PUT FIGURES HERE SHOWING TRAJECTORIES FOR SIMPLE JOURNEYS.

\section{Numerical implementation notes}\label{sec:implementation}

HOGG: Newton's method (CITE), invert, solve, etc.
Note that in practice, Newton's method works amazingly well.
For Newton's method we make use of the derivatives
\begin{align}
    \frac{\dd\vec{F}_\air}{\dd\vboat} &= -\frac{\rho_\air}{|\Delta\vair|}\,A_\air\,\Delta\vair\,\Delta\vair^\top-\rho_\air\,|\Delta\vair|\,A_\air
    \\
    \frac{\dd\vec{F}_\water}{\dd\vboat} &= -\frac{\rho_\water}{|\Delta\vwater|}\,A_\water\,\Delta\vwater\,\Delta\vwater^\top-\rho_\water\,|\Delta\vwater|\,A_\water
    ~,
\end{align}
which are order-2 tensors.

Look-up tables for proper boat trim.

Path planning.

\section{Discussion}\label{sec:discussion}

HOGG: What did we find? What's important about that?

HOGG: Galilean symmetry, coordinate freedom?

HOGG: What were our assumptions? What are the myriad ways in which those assumptions must be wrong? For example, boat has an asymmetry in the wind too, so really the boat must also matter. Viscosity; why can we ignore it? But turbulence; we can't ignore that!

HOGG: Take some time to bash the Bernoulli bullshit.

HOGG: All code used to make the figures in this \documentname{} are available HOGG WHERE?

\paragraph{Acknowledgements:}
It is a pleasure to thank Hans-Walter Rix (MPIA) for valuable discussions.

\raggedright
\printbibliography
\end{document}
