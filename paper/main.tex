% This paper is copyright 2021 the authors.

% people who might comment on it:
% - Rix
% - KSF
% - Kilian Walsh

\documentclass{article}
\usepackage[utf8]{inputenc}

% math macros
\usepackage{amsmath, amsfonts}
\DeclareMathOperator*{\argmax}{argmax}
\newcommand{\dd}{\mathrm{d}}
\renewcommand{\vec}[1]{\boldsymbol{#1}}
\newcommand{\uvec}{\vec{\hat{e}}}
\newcommand{\tensor}[1]{\mathbf{#1}}
\newcommand{\air}{\text{air}}
\newcommand{\water}{\text{wat}}
\newcommand{\boat}{\text{boat}}
\newcommand{\destination}{\text{dest}}
\newcommand{\sail}{\text{s}}
\newcommand{\keel}{\text{k}}
\newcommand{\vair}{\vec{v}_\air}
\newcommand{\vwater}{\vec{v}_\water}
\newcommand{\vboat}{\vec{v}_\boat}

% text macros
\usepackage{biblatex}
\addbibresource{sailing.bib}
\newcommand{\documentname}{\textsl{Article}}
\newcommand{\secref}[1]{Section~\ref{#1}}

% typesetting adjustments
\makeatletter
\renewcommand\section{\@startsection {section}{1}{\z@}%
  {-3.25ex \@plus -1ex \@minus -.2ex}%
  {1.5ex \@plus .2ex}%
  {\raggedright\normalfont\large\bfseries}}
\makeatother
\addtolength{\textheight}{2.20in}
\addtolength{\topmargin}{-1.00in}
\linespread{1.08}
\frenchspacing\sloppy\sloppypar\raggedbottom

\title{\bfseries%
A simple, anisotropic ram-pressure model for sailing}
\author{DWH, MK, others?}
\date{June 2021}
\begin{document}
\maketitle

\begin{abstract}\noindent
    We present a simplified model for the forces on a sailboat.
    Our model is that there is a ram-pressure force on the sail, with a magnitude and direction that depend on the alignment of the relative air--boat velocity with the sail orientation according to an anisotropic drag tensor.
    Similarly, there is a ram-pressure force on the keel, depending on the relative water--boat velocity.
    The resulting expression for the total force on the boat is coordinate-free and consistent with Galilean relativity.
    The solutions corresponding to sailing in steady wind correspond to boat velocities that deliver vanishing total force, given the angles of the sail and keel.
    We use this model to derive simple rules for sailing that are not far off from standard sailing practice, for sailing in different directions (including upwind).
    One consequence of this model is that a sailboat does not move (relative to the water) precisely in the direction that its keel points (the direction of the prow); it moves slightly downwind of that.
    Another consequence is that it is possible for an extremely anisotropic (extremely well designed) boat to sail downwind (relative to the water) \emph{faster than the wind}, on a tacking course.
    Interestingly, nothing about the mechanical model or discussion depend on the \emph{curvature} of the sail, in contradiction to some folk explanations of sailing.
\end{abstract}

\section{Introduction}\label{sec:intro}

KLEBAN: Write things about ddwfttw, maybe?

HOGG: Stuff about Bernoulli and flying and aerofoils that turned out to be wrong; a motivation to reconsider sailing too.

\section{Ram-pressure sailboat}\label{sec:boat}

We are going to work in 2D, where velocities $\vec{v}$ are column vectors (or $\vec{v}\in\mathbb{R}^{2\times 1}$).
To stay coordinate-free, we are going to talk about angles as orientations of unit vectors like $\uvec_{\perp\sail}$ and $\uvec_{\parallel\sail}$ (which will be defined below).
If you prefer to think in terms of angles, you can choose a coordinate system and then
\begin{align}
    \uvec_{\perp,\sail} &= \begin{bmatrix}\cos\theta_\sail \\ \sin\theta_\sail\end{bmatrix} ~ ; ~ \uvec_{\parallel,\sail} = \begin{bmatrix}-\sin\theta_\sail \\ \cos\theta_\sail\end{bmatrix} ~,
\end{align}
for some angle $\theta_\sail$ in that coordinate system.
We won't particularly use angles much in what follows, but if you want to come back to angles, you can take the arctangent of the components of $\uvec_{\perp\sail}$.

Our sailboat is going to be an anisotropic object such that wind that is perpendicular to the sail will produce more ram pressure than wind at any other angle (and far more than wind that ``goes along'' the sail direction).
In detail, the ram-pressure force from the air on the sailboat will be
\begin{align}
    \Delta\vair &\equiv \vair-\vboat
    \\
    \vec{F}_\air &= \rho_\air\,|\Delta\vair|\,\tensor{A}_\air\,\Delta\vair \label{eq:Fair}
    \\
    \tensor{A}_\air &= A_{\perp\sail}\,\uvec_{\perp\sail}\,\uvec_{\perp\sail}^\top + A_{\parallel\sail},\uvec_{\parallel\sail}\,\uvec_{\parallel\sail}^\top \label{eq:Aair}
    \\
    A_{\perp\sail} &\gg A_{\parallel\sail} \label{eq:sail}
    ~,
\end{align}
where $\rho_\air$ is the mass density of air, $\vair,\vboat$ are the velocities of the air and boat, $\tensor{A}_\air$ is a Hermitian tensor with units of area, $A_{\perp\sail}, A_{\parallel\sail}$ are two constants with units of area defining the geometry of the sail, $\uvec_{\perp\sail}$ is a unit vector pointing normal to the surface of the (assumed planar) sail, and $\uvec_{\parallel\sail}$ is a unit vector perpendicular to that.
If you are uncomfortable with the expression for the tensor $\tensor{A}_\air$ in \eqref{eq:Aair}, look at the discussion of tensors in, for example, \cite{kusse}.
The inequality \eqref{eq:sail} says (loosely) that the sail (plus hull) is much bigger face-on than edge-on.
The force from the water is similar
\begin{align}
    \Delta\vwater &\equiv \vwater-\vboat
    \\
    \vec{F}_\water &= \rho_\water\,|\Delta\vwater|\,\tensor{A}_\water\,\Delta\vwater \label{eq:Fwater}
    \\
    \tensor{A}_\water &= A_{\perp\keel}\,\uvec_{\perp\keel}\,\uvec_{\perp\keel}^\top + A_{\parallel\keel},\uvec_{\parallel\keel}\,\uvec_{\parallel\keel}^\top
    \\
    A_{\perp\keel} &\gg A_{\parallel\keel} ~,
\end{align}
where we have gone from air to water and sail ``s'' to keel ``k''.
The ram-pressure forces given in \eqref{eq:Fair} and \eqref{eq:Fwater} can be justified with a simple molecular model in which air and water molecules are elastically bouncing off the sail and keel surfaces; this is the same set of assumptions that leads to, for example, the ideal gas law (HOGG CITE).

Mathematically, steady sailing in steady wind and steady current (constant $\vair, \vwater$) is given by
\begin{align}\label{eq:sailing}
    \vec{F} &= \vec{F}_\air + \vec{F}_\water = \vec{0} ~.
\end{align}
If we want to operationalize steady sailing in code, steady sailing can be thought of as a root-finding problem, in which we find the vector $\vboat$ that makes \eqref{eq:sailing} true, given the orientations $\uvec_{\perp\sail}, \uvec_{\perp\keel}$ of the sail and keel. KLEBAN CHECK THIS: This is a quartic root-finding problem so it has no simple closed form; we solve it numerically in the code associated with this \documentname\footnote{HOGG GIVE LINK HERE} by means of the vector generalization of Newton's method.
For Newton's method we make use of the derivatives
\begin{align}
    \frac{\dd\vec{F}_\air}{\dd\vboat} &= -\frac{\rho_\air}{|\Delta\vair|}\,A_\air\,\Delta\vair\,\Delta\vair^\top-\rho_\air\,|\Delta\vair|\,A_\air
    \\
    \frac{\dd\vec{F}_\water}{\dd\vboat} &= -\frac{\rho_\water}{|\Delta\vwater|}\,A_\water\,\Delta\vwater\,\Delta\vwater^\top-\rho_\water\,|\Delta\vwater|\,A_\water
    ~,
\end{align}
which are order-2 tensors.

HOGG: PUT FIGURES HERE SHOWING THE BEHAVIOR OF THE BOAT IN THE WIND.

HOGG: Comments: Don't sail DIRECTLY in the direction the keel is pointed. Many of these settings sail backwards: We don't distinguish forwards and backwards! Our boat is quadrupolar!

\section{Good sailing}\label{sec:sailing}

In \secref{sec:boat} we showed how the boat moves, given any arbitrary setting of the orientation of the keel and the sail.
However, one doesn't sail by randomly setting the keel and sail!
One sails by pointing the boat some direction and setting the sail (or sails) appropriately.
In our simple, ram-pressure boat, how should we set the planar sail?

HOGG:...

\section{Competitive sailing}\label{sec:racing}

A type-A sailor---or maybe a competetive sailor---is trying to get from a current position to a destination as quickly as possible.\footnote{We don't particularly endorse this attitude towards sailing. The journey \emph{is} the destination.}
Our view is that this sailor should start by specifying the unit vector $\uvec_\destination$ that points from the current position of the boat towards the destination of the boat.
The sailor should then maximize the scalar product $\uvec_\destination^\top\,\vboat$
\begin{align}
    \uvec_{\perp,\sail}^\ast,\uvec_{\perp\keel}^\ast &\leftarrow \argmax_{\uvec_{\perp,\sail},\uvec_{\perp\keel}} (\uvec_\destination^\top\,\vboat) ~,
\end{align}
where $\uvec_{\perp,\sail}^\ast,\uvec_{\perp\keel}^\ast$ are the best settings of the orientations of the keel and sail, given the direction $\uvec_\destination$ to the destination, and implicitly $\vboat$ is being thought of as a function of these orientations.
Note that the sailor explicitly \emph{does not want} to travel directly towards the destination:
The sailor wants to make as much progress as possible in that direction, but is willing to tack back-and-forth to get there, if it gets the boat there faster.
So the metric really is the projection of the boat velocity onto the displacement vector pointing from the current position of the boat to the destination.

While this is simple to state, the numerical optimization hurts...HOGG

HOGG PUT FIGURES HERE SHOWING THE OPTIMAL SETTINGS OF SAIL AND KEEL ORIENTATIONS FOR DIFFERENT DESTINATION DIRECTIONS.

Now if you want to plan a complete path or journey for the sailboat in (unrealistically) steady wind, you... HOGG

HOGG PUT FIGURES HERE SHOWING TRAJECTORIES FOR SIMPLE JOURNEYS.

\section{Numerical implementation notes}\label{sec:implementation}

Newton's method, invert, solve, etc.

Look-up tables for proper boat trim.

Path planning.

\section{Discussion}\label{sec:discussion}

HOGG: What did we find? What's important about that?

HOGG: Galilean symmetry, coordinate freedom?

HOGG: What were our assumptions? What are the myriad ways in which those assumptions must be wrong? For example, boat has an asymmetry in the wind too, so really the boat must also matter. Viscosity; why can we ignore it? But turbulence; we can't ignore that!

HOGG: Take some time to bash the Bernoulli bullshit.

HOGG: All code used to make the figures in this \documentname{} are available HOGG WHERE?

\paragraph{Acknowledgements:}
It is a pleasure to thank Hans-Walter Rix (MPIA) for valuable discussions.

\raggedright
\printbibliography
\end{document}
